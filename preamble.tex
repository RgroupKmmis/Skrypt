\usepackage{multirow}
\usepackage[table,xcdraw]{xcolor}
\usepackage{amssymb}
\usepackage{natbib}
\usepackage{tabularx} % in the preamble
 \linespread{1.5} 
\usepackage{tabulary}
\usepackage{float}
\usepackage{etoolbox}
\apptocmd{\thebibliography}{\csname phantomsection\endcsname\addcontentsline{toc}{chapter}{\bibname}}{}{}
% \usepackage[MeX,OT4,plmath]{polski}
\usepackage[utf8]{inputenc} % albo latin2, cp1250
%\usepackage[OT4]{fontenc}
\usepackage[polish]{babel}
\usepackage{graphicx} % wstawianie grafiki
\usepackage{tikz} 
\usepackage{subfig}  %zeby obrazy byly obok siebie
\usepackage{booktabs}
\usepackage{amsthm}
\makeatletter
\def\thm@space@setup{%
  \thm@preskip=8pt plus 2pt minus 4pt
  \thm@postskip=\thm@preskip
}
\makeatother

